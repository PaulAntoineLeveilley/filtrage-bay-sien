\documentclass{article}
\usepackage[utf8]{inputenc}
\usepackage{geometry}
\geometry{hmargin=2.5cm,vmargin=2.5cm}
\usepackage{mathtools} %\usepackage{align}
%\usepackage[french]{babel}

\title{SOD333 - Rapport}
\author{Paul-Antoine Leveilley \& Mila Rocco}
\date{September 2022}

\begin{document}

\maketitle

\begin{center}
Chaque rapport de TP doit faire environ 5 pages.
\end{center}







\newpage
\section{TP1 : titre du TP}

\subsection{Introduction}
On étudie la loi d'espérance µ définie ci-dessous, où la fonction de densité q de l'échantillon étudié est inconnue :
$$\mu = \int_0^1 g(x)q(x)dx = \int_0^1 cos(\frac{\pi x}{2})dx\ (=\frac{2}{\pi})$$

 \subsection{Application "brute" de la Méthode de Monte Carlo}
On décide d'étudier l'échantillon $(X_i)_{1\leq i \leq N}$ , qui suit la loi uniforme sur $[0;1]$ afin de vérifier qu'on estime bien µ avec la méthode de Monte Carlo. Sa fonction de densité est donc $q=U([0;1])$. On choisira $N=50$ pour l'étude empirique du problème.

Pour évaluer µ, on applique la méthode de Monte Carlo, et on obtient l'approximation 
$$\hat{\mu}_N = \frac{1}{N} \sum_{i=1}^N g(X_i)$$

- Calculer la variance théorique\\
\begin{align*} 
  Var(\hat{\mu}_N) &= \frac1{N^2} \sum_{i=1}^N Var(g(X_i))\\ 
  &= \frac1{N^2}\times N Var(cos(\frac{\pi X}2)) \\ 
  &= \frac1{N} \left ( \int_0^1 cos^2(\frac{\pi x}2)dx - \left ( \int_0^1 cos(\frac{\pi x}2)dx \right )^2 \right )\\
  &= \frac1{N} \left ( \int_0^1 \frac{1+cos(\pi x)}{2}dx - 
   \frac4{\pi^2} \right )\\
  &= \frac1{N} \left ( \frac{1}{2} - 
   \frac4{\pi^2} \right )\\
\end{align*}

- Estimer empiriquement la variance (prendre N = 50)\\
\textbf{résultats du TP.}
On applique la méthode de Monte Carlo et on en tire un échantillon de $NbMC=1000$ tirages.

\subsection{Echantillonage pondéré}
On génère un nouvel échantillon $(X_i)_{1\leq i \leq N}$ suivant maintenant une fonction d'importance (FI) $\Tilde{q}$ au plus proche de $g(x)$: $X_i \hookrightarrow \Tilde{q}$. Le changement de probabilité donne : 
$$ \hat{\mu}_N = \frac1{N} \sum_{i=1}^N g(X_i)\frac{q(X_i)}{\Tilde{q}(X_i)}\ \overset{p.s.}{\longrightarrow}\ \mu $$

- Chercher une bonne FI\\
(idée : DL à l'ordre 2 au voisinage de 0)  \\
On approxime la fonction g par son développement à l'ordre 2 en 0. Rappelons le développement limité en 0 de la fonction cosinus: 
$ cos(x) = 1 - \frac{x^2}{2} + o(x^2) $.
On prend donc pour Fonction d'Importance : 
$$\Tilde{q}(x):= 1 - \frac{\pi^2}{8} x^2$$
$$(Prof :\ \Tilde{q}(x):= \frac32(1 - x^2)$$

- Calculer la variance théorique\\
\textbf{Application numérique (TP) :}

- Utiliser la méthode de rejet pour  générer suivant la FI \\ (Comparer la probabilité d’acceptation théorique à celle obtenue par simulations)\\
On génère un échantillon suivant $p$
$$p(x) = \frac{g(x)q(x)}{\int_0^1 g(x)q(x)dx}$$

- Estimer empiriquement la variance.\\



\subsection{Conclusion}


















\newpage
\section{TP2 : titre du TP}
\subsection{Introduction}
\subsection{Titre intermédiaire}
\subsection{Conclusion}

\newpage
\section{TP3 : titre du TP}
\subsection{Introduction}
\subsection{Titre intermédiaire}
\subsection{Conclusion}

\newpage
\section{TP4 : titre du TP}
\subsection{Introduction}
\subsection{Titre intermédiaire}
\subsection{Conclusion}

\end{document}
